\documentstyle[12pt]{article}

\begin{document}

\section{Multiple Data Sources}

\section{General}

In MiniBooNE, data will come from several separate sources.  At the time of 
this writing, these are:
\begin{itemize}
\item The Tank - Primarily the data from the BooNE detector, but also some 
miscellaneous data read out with the same electronics, and trigger information.
\item ACNET - all data from the accelerator data acquisition system which is
associated with MiniBooNE operation.  This data will be come from an XML
server in a well-defined format.  A dedicated computer will read the data
of interest to MiniBooNE.
\item Resistive Wall Monitor (RWM) - this is the monitor for the beam time
profile (bunch structure) just prior to the target. The data acquisition
engine will be a VXI-based 1 GHz oscilloscope, which is
read out with a Linux platform computer.
\item Little Muon Counter (LMC) - the detector to monitor the
small angle muons flux in the decay region, as a measure of the
kaon content of our beam.  Data will be read out from a dedicated
data acquisition computer.
\end{itemize}

In general, all subsystems will read out data every 1.6 $\mu $s spill, and 
it will be necessary to merge this data for analysis.  In addition, some 
subsystems will read out data in between spills, which may be treated
as asynchronous.

The plan is that each subsystem will run a more or less standalone
data acquisition system, which writes data to locally accessible file.
All data will be GPS timestamped for use in subsequent merging. Because
we are only interested in merging data from a particular spill, and
spills are 67 ms apart, a timestamp accuracy of 1 ms or so is more
than sufficient.

The individual data acquisition systems will be controlled by the
``\"{U}berDAQ'' to insure that the data from all the systems has the
same run boundaries; however, sub-run boundaries may vary from system
to system.

\section{Run Control}

Operationally, the integrity of the tank data takes 
precedence over everything else.  For that reason, the \"UberDAQ
will send a send control commands to the tank DAQ program, and
only when it receives an acknowlegement will it proceed to start
up the other systems.

All other systems are responsible for implementing the following
commands:

\begin{itemize} 
\item prepare - Initialize all systems and begin reading and buffering data
if appropriate.
\item start_run (number) (timestamp) - Open a file for run (number) and
begin writing all data after GPS time (timestamp).  Note, if a run is
already running, that run should be stopped at (timestamp) and the new
run started.  This is referred to a ``run hot swapping''.
\item pause_run (timestamp) - pause  run at GPS (timestamp).
\item resume_run (timestamp) - restart run at GPS (timestamp).
\item stop_run (timestamp) - Stop current run at GPS (timestamp) and close
file.
\item shutdown - Stop data buffering and kill any particularly
resource intensive processes.  Subsequent runs must be preceeded
by a ``prepare'' command.
\end{itemize}

In general, subsystems should allow for (timestamp) to be 
up to a few seconds {\em either before or after} the current time,
so in most cases, it will be necessary to establish a data pipeline.



\end{document}



