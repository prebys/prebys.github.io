\documentclass{article}
\usepackage {epsfig}

\title {Accelerator Improvement Options for NuMI Proton Intensity}
\date {July 18, 2002}
\author {The NuMI Proton Intensity Working Group\\
  B. Choudhary, T. Fields, J. Griffin, P. Lucas, A. Marchionni,\\
P. Martin (co-chair), D. Michael(co-chair), E. Prebys, S. Pruss}


\begin {document}
\maketitle
\abstract {In order to meet the needs for protons for MINOS and other
experiments, substantial improvements will be necessary in the Booster 
and Main Injector. We have evaluated a number of 
improvements which would yield
an increase in the number of protons on target for NuMI and
Mini-BooNE in the years 2005-2008. We outline a possible program of
improvements in the Booster and Main Injector which can be implemented
in steps
over a five year period and which could result in an increase in proton
intensity through the Main Injector which is approximately four times
what is currently possible. We provide a list of specific improvements
and suggest a possible schedule for the implementation. 
}

\section {Introduction}

  The rate at which statistics can be accumulated in many experiments scales
directly with the number of protons which can be accelerated to the
appropriate energy and delivered to that experiment. Proton intensity is a 
crucial issue for
neutrino oscillation experiments. For these experiments, one
typically builds the largest mass of detector which meets the experimental
requirements 
and can be afforded. For world-class experiments, not only must the
detector be very massive (and expensive) but the source of protons must be
very intense. The Fermilab Main Injector is well suited to become a world-class
proton facility for neutrino beams. However, the capabilities
of the Main Injector will be stressed by the demands from the coming round
of neutrino oscillation experiments. Upgrades to the accelerator complex
will be essential.

In this report, we evaluate the current state of the ability of the Fermilab
complex to deliver protons, extrapolate to the 2005
experimental program (Collider, MINOS and Mini-BooNE) and make some
projections regarding the longer-term future of the complex and proton
economics prior to the commissioning of a possible new proton driver. 
In some cases,
we anticipate that upgrades to the Main Injector which will also be essential
for a new proton driver and will already start to deliver additional proton
intensity even before the proton driver itself will be commissioned. We
believe that such upgrades present a highly attractive and cost-effective 
investment path for the laboratory.

\section {Proton Economics for MINOS and Mini-BooNE}
\subsection {The Current and Near-Term Situation}
\label {sec:current}

Up to now, the Main Injector has run primarily
for antiproton production for the collider. In the current mode of operation,
a single batch of $\approx 4.5\times 10^{12}$ protons is first accelerated
in the Booster to 8 GeV and then injected into the Main Injector and
accelerated to 120 GeV before being delivered to the antiproton production
target. The current cycle for the Main Injector is 2.46 seconds, determined by
the cycle time for Pbar source. The resulting
number of protons accelerated per year is about $3-4\times 10^{19}$, in both
the Booster and Main Injector. In the most recent six month period
a total of $0.82\times 10^{19}$ protons were actually accelerated through
the Main Injector for anti-proton production. This accentuates the point that
for production to be at its peak value that the complex must work together
as a whole. 

Recently, the demand for protons
accelerated in the Booster has gone up dramatically with the commissioning
of Mini-BooNE. The current request is to run the Booster at
5 Hz acceleration cycles for Mini-BooNE whenever it isn't being used for
filling the Main Injector. Improvements are required in the extraction
septum and power supply and also reduction in proton losses are necessary
in order to achieve this rate for Mini-BooNE.


In early 2005, MINOS will also begin to run and place yet more
demands on both the Booster and Main Injector. MINOS requires that the 
Main Injector run in ``Multi-Batch'' mode where six batches of protons are
injected from the Booster in every MI cycle. At the same time, because of 
physics demand for Mini-BooNE to operate in anti-neutrino mode as well as
neutrino mode, it is likely that both MINOS and Mini-BooNE 
will run simultaneously
placing additional demands on the Booster. We note that the 
Fermilab baseline plan (1998) for MINOS
running calls for $3.8\times10^{20}$ protons per year 
delivered to the NuMI target \cite {NuMITDR, Peoples}, 
roughly 10 times the number of protons 
currently accelerated to 120 GeV. Also by 2005, it is expected that the
number of 120 GeV protons delivered for
anti-proton production will double by use of some type of stacking scheme
\cite {RunIIB}.
Given these demands, the total number of protons which will have to
be accelerated through the Booster will have to be roughly 20 times what 
has routinely been accelerated. 

One additional consideration is that the laboratory
is currently building the ability to extract Main Injector beams for
test beams and experiments (one of which is already approved, E907) in the
Meson area. These experiments will not substantially increase the demand for
protons, but they will affect the integrated proton intensity  by requiring 
an extended Main Injector cycle (flat top) for slow extraction.

The first step in moving towards the future has already been taken in 
preparations for the intial running of Mini-BooNE. Upgrades to the
shielding around the Booster have been added at some locations in order to
ensure that external radiation limits will not reduce the number of protons
which can be accelerated. In addition, beam ``notching'' has been introduced
to reduce the exposure of critical devices to radiation at extraction.
Even with these improvements, the number of protons which can be delivered
to Mini-BooNE over the next year or so will be limited not by any intrinsic
features of the Booster but rather by proton losses causing the machine to
become too radioactive. Work is underway to improve that situation by adding
additional RF controls and strategic collimation where protons will be lost
rather than in critical devices such as RF cavities. In order for Mini-BooNE
and MINOS to run simultaneously, yet another factor of two improvement will be
needed in the loss of protons compared to what is expected from current Booster
improvement projects.

Once MINOS begins running, the Main Injector must run in ``multi-batch'' mode.
Currently, high-intensity, 
multi-batch mode is in a distinctly developmental status. 
Multi-batch operation of the Main Injector was briefly demonstrated when it
was first commissioned in 1999. For that demonstration, six batches
containing a total of $2 \times 10^{13}$ protons were accelerated to 120 GeV.
For a variety of technical reasons, it is not currently possible to replicate
this intensity, but with some investment in the complex it is anticipated
that this should certainly be feasible again within the next year or so.
In December of 2001, the study of  multi-batch acceleration was once again
undertaken by a group interested in studying and developing the capabilities 
for MINOS. The current ability to accelerate protons in multi-batch mode is
limited to about $1.5 \times 10^{13}$ protons per cycle. Main Injector experts
believe that improvements in the RF feedback and damping will permit
multi-batch operation to accelerate up to $3 \times 10^{13}$ protons per
cycle by 2005 with one (of six) of those batches going to antiproton production
and the remaining five going to NuMI/MINOS. Note that achieving even this
intensity of protons requires that the average number of protons in a Booster
batch must be $5 \times 10^{12}$. 

\subsection {The MINOS+Collider Era, 2005 to 2008}

Starting in 2005, the Fermilab accelerator complex will face an unprecendented
demand for numbers of protons. 
Current Beams Division planning assumes the following
operating scenario for 2005:
\begin {itemize}
\item Booster operation will be improved so that a total of $5\times 10^{12}$
protons will be delivered per batch with acceptable losses so that
$10^{21}$ protons per year can be accelerated.
\item Main Injector operation will be improved so that a total of six
batches (each with $5\times 10^{12}$ protons) can be reliably accelerated
with a cycle time of 1.9 seconds. Note that this already assumes that the pbar
stacking rate is improved from the current 2.5 s.
There is some ambiguity regarding
the impact of slip-stacking for Run IIB which could increase the cycle time.
\item A total uptime of $1.8\times10^7$ seconds of production acceleration
cycle per year will be realized for a total of $\approx 2.8 \times 10^{20}$
protons accelerated to 120 GeV per year. 
\item One-sixth of those protons will go to anti-proton production and
five-sixths will go to the NuMI target. It is assumed that
running for other fixed target experiments and/or test beam will impact
the total protons accelerated by no more than 10\%. (This may come in
a number of different forms.) 
\end {itemize}

Although the above scenario appears realistic, we note that the program is
already going to be short of nominal expectations and anticipate that the 
following issues need to be considered:
\begin {enumerate}
\item Under the above scenario, the number of protons delivered to the NuMI
target will be only $2.4 \times 10^{20}$ per year rather than the Fermilab
original design plan of $3.8 \times 10^{20}$ per year.
\item As mentioned, a preliminary plan exists for slip-stacking one batch of 
protons into
the Main Injector to increase the anti-proton production \cite {RunIIB}. 
The slip-stacking
will increase the cycle time of the Main Injector once implemented, 
nominally reducing the number of protons delivered to the NuMI target 
by ~10\%. It is unclear whether it is possible to slip-stack additional
batches for NuMI. It may be possible to slip stack a total of six batches
(two for pbar and four for NuMI) and then add another 3 batches for NuMI,
but this will increase the cycle time of the Main Injector so that the
gain will be relatively small (if any) for NuMI and the relative protons
for pbar production will go down compared to slip-stacking just two batches.
We believe that stacking in the Main Injector is a very serious issue for
the laboratory to consider. An alternative to slip-stacking has been 
studied, barrier RF stacking. This has the advantage that it may permit more
beam to be stacked in the same time (or less) than slip stacking.
We believe that barrier RF stacking must be
considered with high priority.
\item Another part of the planning for RunIIB is that $\bar p$'s will be
transferred once every 15 minutes from the accumulator to the recycler via
the Main Injector. It is estimated that it will take about one minute of
the Main Injector for this operation (although currently it takes about
one hour). If we assume that it takes one minute of every 15 that will result
in an 7\% additional loss of protons for NuMI.
\item Simultaneous MINOS+Mini-BooNE running. Either a way to accelerate more
protons in the Booster per hour 
by a factor of two must be developed or running MINOS
and Mini-BooNE simultaneously will reduce the proton intensity to MINOS
by a factor of two. 
\item Should a substantial test-beam program be undertaken, one may anticipate
an additional 10\% reduction in protons delivered to NuMI.
\item We note that actual accelerator complex ``up times'' may not quite 
meet the expectation due to a variety of reasons. Although
the recent performance, as noted in section \ref {sec:current} is due
somewhat to the fact that the maximum possible production was not
yet attempted, some is also the result from ``unexpected'' downtime.
During NuMI running additional losses
at the 10\% level due to complex downtime would not be too surprising. 
\end {enumerate}

Hence, we observe that in order to meet the overall needs for protons
during 2005-2008
that a factor of two improvement will be needed in the time-averaged
number of protons
which can be accelerated to 8 GeV in the Booster compared to any current
plan and that a separate
factor of at least two improvement will also be required in the
ability to accelerate protons in the Main Injector to 120 GeV compared to any
existing plan. In fact, some of the plans currently being pursued for Run IIB
are almost certain to actually further reduce the number of protons delivered
to the NuMI target. In this time-scale, it is clear that improvement will
not result from a new proton driver. 

\section {Overview of Improvements}

In this section, we present an overview of upgrades to the Booster
and Main Injector which will help to meet the demand for protons in 2005 
and beyond. For each item presented here, there is a corresponding 
section in the appendix which expands on some details. We believe that
all of the improvements listed here can be implemented over the next
few years and we describe a possible time-line for that in the following
section. In some cases,
a specific budget and schedule have already been established by Beams Division
for the listed work. Where we know that to be the case,
 we note that by stating ``already planned''. We note that our working group
is not privy to all of the internal planning in Beams Division so it is
possible that not all instances of work already planned have been so noted
here. 

It is worth noting that all of the improvements listed here have been suggested
by various individuals within Beams Division and much of the technical
information has been supplied by relevant system experts as much as possible.
Where they exist, we make an effort to explicitly reference notes previously
written on a number of these topics. We have also circulated this report
to various Beams Division experts in order to solicit their advice and 
feedback on the technical issues presented. The suggestions and advice of
many members of Beams Division have been invaluable in producing this 
report. This input also leads us to believe that the upgrades which we
present here have a good probability of providing the suggested increases
in proton intensity. That said, we realize that the aggregate increase
in intensity will likely be somewhat poorer than the nominal increase
one might expect simply by multiplying many factors together. The aggregate
increases which we present in section \ref {sec:timeline} 
take some account of a ``reality factor''. 

\subsection {Booster Improvements}

Proton acceleration in the Booster is currently limited by the rate at
which proton acceleration cycles can be executed and by
beam instabilities at high intensities leading to radiation from proton losses
and extracted beam with poor properties for transfer and acceleration in
the Main Injector \cite {Webber}. Booster
upgrades can both directly increase intensity and make for higher MI 
intensity by providing clean beams for stacking and easier acceleration.

An additional important improvement is in the acceleration cycle time. 
In any scenario, the Booster acceleration cycle rate ($\le 15$ Hz
magnet cycle rate) contributes significantly to the cycle time for the
Main Injector. The limitations are primarily on the average acceleration
cycle rate. For example, if six batches are accelerated in the Booster
in a 400 ms ``burst'' every 1.9 seconds (for filling the Main Injector)
the corresponding average acceleration cycle rate is 4.2 Hz. (We note that
two ``pre-pulses'' are executed on the extraction magnet systems prior to
such acceleration cycles.)
If we add
ten additional accleration cycles for Mini-BooNE in the ``extra'' 1.5 seconds
then the average accleration cycle rate is 9.5 Hz. If we then add
another 6 cycles for MI barrier stacking (stretching the MI cycle time
to 2.3 s) the average rate climbs to
10.4 Hz. If the Main Injector cycle time is lowered to 1.0 s then the
maximum possible Booster cycle rate of 15 Hz will be demanded if both
Mini-BooNE and MINOS are to run simultaneously and even 12 Hz will be 
required just for loading batches into the Main Injector with barrier
stacking.

Loss of protons creates radioactivity both at the surface and activates
machine components. The most critical components with respect to activation
are the RF cavities. Current proton loss rates result in activation at
these locations that results in typically 50-250 mrem/h at a distance of
1 ft from the cavities. This is high enough that it limits the number of
protons which can be accelerated to about $3\times10^{15}$ per hour
\cite {Webber3}. It is critical for future operations that the dose to the
RF cavities be reduced through both better beam control and collimation.

Limitations in the direct intensity increase are not completely understood,
but many experts estimate that it should be possible to increase intensity
from the current typical operation, $\approx 4.5\times10^{12}$ protons
per batch, to $\approx 6-7\times10^{12}$ protons per batch. The cost of
most of these systems for the Booster are very modest compared to the
cost of a new proton driver, but the returns are correspondingly modest,
though much needed.
   \begin {itemize}
   \item Hardware upgrades to permit higher acceleration cycle rates. 
         The Booster magnets cycle at 15 Hz but not all of the components
         for acceleration can cycle at that average rate. Although the 
         instantaneous acceleration cycle rate can be 15 Hz, the current 
         components require a lower average acceleration cycle rate.
         to cycle that fast. The current average accleration cycle rate 
         is limited to about 3 Hz.
         Several upgrades are already in progress:
         \begin {enumerate}
         \item New extraction septum power supply: should permit 5 Hz. 
               Ready soon.
         \item New extraction septum magnet: should permit 7.5 Hz. In
               fabrication.  Note that at this point, the rate is limited
               by cable heating.  Adding additional cable penetrations would
               allow the extraction septum to be pulsed at the full
               15 Hz rate.
         \item As discussed above, additional rate increases, beyond those
               already planned are very attractive. Although the existing
               RF system is nominally designed to operate at 15 Hz, its
               reliability in such a mode is not known and may require
               substantial upgrades \cite {Webber2}. We recommend that 
               upgrades with rate capability approaching 15 Hz should be
               undertaken (this may be staged over several years).
         \end {enumerate}
   \item New hardware to help stabilize the beam and reduce proton
         losses including:
         \begin {enumerate}
         \item Improved longitudinal damping.
         \item Ramped correctors (installed).
         \item New RF damping hardware.
         \item New collimators  (installed but not yet shielded).
         \item Resonant halo extraction during acceleration (being studied).
         \item Cogging and notching capabilities to limit beam losses during
               extraction (already planned).
         \item Larger aperture RF cavities (already planned).
         \item Inductive inserts.
         \item Additional acceleration RF and controls to permit the beam
               to be stretched out, reducing losses due to space-charge.
         \end {enumerate}
   \end {itemize}
We note that although budget has been planned for some of the above activities
that manpower resources have not yet been assigned in some cases. It is 
important that work begin soon on many of these activities in order to avoid
serious proton shortfalls in 2005.

\subsection {Main Injector Improvements}
\label {sec:MIimprove}

Improvements in the Main Injector fall into three
      main categories; decreasing the cycle time, improvements to permit
      the machine to accelerate more protons per cycle, 
      and proton stacking injection schemes. 
      As the total number of protons in the
      machine is increased, it will be necessary to add some extra RF power and
      damping under any circumstance, and this will be particularly true
      with proton intensities that could become available with a new proton 
      source. Hence, much of that investment can be viewed as ``on the path'' 
      of a new proton source.  

Although outside the scope of this report, we note that the pbar production
rate can have an impact on the intensity of NuMI protons. 
Unless the Main Injector can ramp roughly twice 
as fast as the pbar source can cycle, it is likely that the MI cycle rate
will be tied to the pbar cycle rate during collider operation (most of
the time for the operations under discussion in this note). Hence, it
is essential that effort is invested to bring the pbar cycle time as low
as possible over a period of several years, hopefully matching the rate 
at which it is possible to cycle the MI in multi-batch mode. Since this is
of importance to the collider run we expect that this is the plan of everyone
involved and we certainly expect this to happen. Here, we simply wish to
point out that it is relevant to NuMI, even though it nominally would seem
not connected. 

A more serious issue for NuMI proton intensity, and one where we see a 
potential conflict between NuMI and the collider program is the plan for using
slip-stacking to increase the
proton intensity to pbar. It is not clear that slip stacking will provide
any benefit for NuMI, and it may reduce the protons available for NuMI
by as much as 10\% if implemented only for pbar production.
We believe that the option of Barrier RF stacking, discussed
in section \ref {sec:barrierRFstack} and references \cite {Griffin,Ng}, 
may provide a better
path for the laboratory to pursue to maximize the physics
potential of all of its experiments in this time frame.

   \begin {itemize}
   \item Additional RF power to handle extra proton intensity. Note that
         this may also contribute to reduction in cycle time. 
   \item Reduction in cycle time (from the nominal 1.9 s which is planned) 
         by ``tuning'' the acceleration cycle. This
         probably would require very little new hardware and could result in
         a reduction in cycle time of 5-10\%. (Note that all reductions
         must be quantized in 67 ms steps in order to be effective due
         to the Booster cycle time.)
   \item Reduction in cycle time by increasing magnet power.
         It is possible to increase the power supplied to the MI magnets,
         reducing the total MI cycle time to as little as 1.0 s. Some
         relatively small investment is needed to bring stable operation
         at 1.5 s cycle time. (Note that this corresponds to the 1.9 s cycle
         time used for the nominal multi-batch operation when one takes into
         account the time at 8 GeV while the Booster injects six batches of
         protons.)
   \item New RF damper electronics and components. Necessary to go to higher
         intensity and more sophisticated and expensive as the intensity goes
         ever higher.
   \item Collimators to protect critical components from beam losses.
   \item Yet more RF power with cavity modifications and/or new cavities 
         coupled with significant additional new magnet power 
         to reduce the MI cycle time down to 1.0s.
   \item Slip stacking: May not work well for multi-batch operation
         due to the long time required to slip the beams and technical 
         difficulties in multi-batch operation. However, should a single
         batch be slip-stacked for the collider this will reduce the protons
         available to NuMI by about 10\%.
   \item Barrier RF stacking: Appears promising for increasing protons
         accelerated to 120 GeV by 60\%. Requires well-behaved Booster
         beam and new barrier RF systems in Main Injector. Operates on
         principles already in use in the Recycler \cite {Recycler}. This
         would simultaneously increase the protons available for pbar 
         production and for NuMI. For the same rate of protons delivered
         to pbar production, using barrier RF stacking could deliver as
         much as a factor of 1.8 times more protons to the NuMI target
         compared to use of slip stacking.
         This appears to be the single most important technical issue
         which we can identify for NuMI proton intensity in the plan
         we present compared to the current Beams Division planning
         \cite {RunIIB}.
   \end {itemize}

\section {A Possible Implementation Schedule}
\label {sec:timeline}

  Implementation of the improvements listed here will require money to invest
in hardware, manpower to study the accelerators and develop and implement
the ideas,
 and time to do the work in a way which is consistent with the ongoing
running of the collider. Hence, it is clear that the suggested program will
need to be staged over a period of a few years. Here, we propose an
implementation schedule which we believe is realistic and consistent with
other Laboratory activities and priorities. 

Note that some of the work
is already a part of the existing planning for the Laboratory. Where that
is the case, we note it by stating ``already planned'' following the task.
However, most ``already planned'' tasks will likely benefit from additional
effort. For each year, we show the expected number of protons which can be 
delivered
to the NuMI target in $1.8 \times 10^7$ seconds of operation
(in earlier years it is clearly hypothetical) in that 
year given that the work for previous years has been completed and assuming
no competing experimental program for protons other than the collider and
Mini-BooNE.

\begin {itemize}
\item {\bf 2002; $1.6 \times 10^{20}$ NuMI protons:}
  \begin {enumerate}
  \item Continue to re-establish multi-batch operation in the Main Injector
        (underway).
  \item Study characteristics of high-intensity MI beam in the accelerator and 
        in Proton 150 (underway).
  \item Implement ramped correcters in Booster (underway).
  \item Start implementing dampers in MI (already planned)
  \item Continue Booster collimation studies (underway).
  \item Begin design of additional Booster RF for reducing space charge.
  \item Start machine studies for barrier RF stacking in Booster, MI
        and Recycler (sort-of planned?).
  \item Start Booster and MI acceleration cycle time reduction studies.
  \item Add notch in Booster for multi-batch operation.
  \end {enumerate}
\item {\bf 2003; $2.4 \times 10^{20}$ NuMI protons:}
  \begin {enumerate}
  \item Continue Booster beam studies. Bring operation to $5.0 \times 10^{12}$
        protons per cycle (already planned?) 
  \item Fabricate and install Booster inductive inserts.
  \item Install and commission prototype large aperture cavity (already
        tested). Start acquisition and fabrication of larger aperture 
        Booster RF cavities (already planned but we wish to emphasize
        the need).
  \item Complete design of additional Booster RF for space charge reduction.
        Start fabrication.
  \item Complete first phase of Booster beam repetition rate increase (bring
        rate to 7 Hz).
  \item Complete implementation of MI dampers (already planned).
  \item Start acquisition of equipment for additional MI RF and magnet power.
  \item Start design of barrier RF stacking components.
  \item Start design of MI collimators to eliminate internal and external
        radiation issues should adiabatic capture studies show these are
        needed. (Otherwise, don't do now.)
  \end {enumerate}
\item {\bf 2004; $2.6 \times 10^{20}$ NuMI protons:}
  \begin {enumerate}
  \item Bring Booster operation to $5.5 \times 10^{12}$ protons per cycle.
  \item Continue fabrication and acquisition for Booster acceleration cycle
        speed-up (bring to 8 Hz).
  \item Install and commission additional Booster RF for space-charge
        reduction.
  \item Continue acquisition and fabrication of large aperture RF cavities
        for the Booster.
  \item Continue fabrication of barrier RF components and start to install 
        and commission.
  \item Continue acquisition of additional MI RF and magnet power. 
        Start installation and commissioning.
  \end {enumerate}
\item {\bf 2005; $3.9 \times 10^{20}$ NuMI protons:}
   \begin {enumerate}
   \item Bring Booster operation to $6.0 \times 10^{12}$ protons per cycle.
   \item Continue fabrication of larger aperture Booster RF cavities. Start
         installing and commissioning.
   \item Continue to increase Booster beam repetition rate (bring to 11 Hz).
   \item Complete acquisition of additional MI RF power. Continue acquisition
      of additional MI magnet power. Reduce MI cycle time
      to 1.70 s.
   \item Complete fabrication of barrier RF stacking components. Start
      to commission stacking. 
   \item Start fabrication of additional MI RF cavitities.
   \end {enumerate}
\item {\bf 2006; $4.9 \times 10^{20}$ NuMI protons:}
   \begin {enumerate}
   \item Continue to increase Booster beam repetition rate (bring to 13 Hz).
   \item Complete fabrication of larger aperture Booster RF cavities. 
         Continue installing and commissioning.
   \item Continue to improve stacking operation by tuning in Booster and
         MI. Fabricate additional control devices as necessary.
   \item Continue acquisition of additional magnet power for MI. Continue
         installation. 
   \end {enumerate}
\item {\bf 2007; $5.1 \times 10^{20}$ NuMI protons:}
   \begin {enumerate}
   \item Complete Booster repetition rate increase program (bring to 15 Hz).
   \item Complete acquisition and installation of magnet power in MI.
   \item Bring MI cycle time to 1.0 s.
   \end {enumerate}
\item {\bf 2008 and beyond; $6.0 \times 10^{20}$ NuMI protons.}
\end {itemize}

We note that under this plan that the integrated number of protons delivered
to the NuMI target in 3 years of running will be 
$\approx 14 \times 10^{20}$.

\section {Approximate Costs and Summary}

\begin {table}
\begin {tabular}{|l|rrrrrr|r|}
\hline
Item & \multicolumn {6}{c}{Costs (\$k) per FY} & Total\\
     & 02 & 03 & 04 & 05 & 06 & 07 & Cost \\
\hline
%{\bf Booster} &&&&&&&  \\
%Extraction septum power supply & 20 &&&&&& 20 \\
%Extraction septum magnet       & 20 &&&&&& 20 \\
%Other duty factor upgrades     &   &  100 & 200 & 800 & 800 & 400 & 2300 \\
%Ramped Correctors              & 20 &&&&&& 20 \\
%Collimators                    & 20 & 20 & &&&& 40 \\
%Cogging and Notching           &    & 20 & 20 &&&& 40 \\
%Larger aperture RF cavities    & 20 & 200 & 2000 & 2000 &&& 4720 \\
%Inductive Inserts              &    & 40 & 60 &&&& 100 \\
%RF for space-charge reduction  &    & 30 & 200 & 20 &&& 250 \\
%\hline
%Total for Booster Upgrades     & 100 & 230 & 800 & 540 & 20 & 20 & 1710 \\
%\hline
%{\bf Main Injector} &&&&&&& \\
%Additional RF power            &   & 100 & 500 & 500 & 500 & 200 & 1800 \\
%Additional magnet power        & & 500 & 1000 & 4500 & 4500 & 4500 & 15000 \\
%Cycle time reduction with tuning & & 20 & 20 &&&& 40 \\
%Dampers                        & 40 & 100 & &&&& 140 \\
%Collimators                    & & & 100 & 100 & 100 && 300\\
%Barrier RF stacking components & 30 & 300 & 300 & 300 & 100 && 1030\\
%\hline
%Total for MI Upgrades          & 40 & 1260 & 1720 & 2700 & 2640 & 1200 & 9560\\
%\hline
%\hline
Spending Profile     & 250 & 1500 & 3000 & 8000 & 8000 & 5000 & 26000 \\
\hline
\end {tabular}
\caption { \label {tab:fundingprof} Suggested funding profile for 
implementation of the upgrades described in this report. Clearly, other
profiles can also be adopted but the profile presented here is based on
what should be technically achievable and is consistent with the implementation
schedule presented here. The costs include purchase of
equipment and the cost of engineers and technicians. 
The costs presented
here are without contingency. We note that given the conceptual status of
many of the proposed upgrades that an overall contingency in the range of 
50-100\% should be assigned. However, because much of the cost is for
purchase of components with already understood costs, we believe that
an average contingency in the range of 50\% is most appropriate.
Physicist manpower
costs are not included but it is assumed that adequate (and substantial)
physicist manpower is available.}
\end {table}

\begin {table}
\begin {tabular}{|l|r|}
\hline
Item & Cost Range (\$k)\\
\hline
{\bf Booster} & \\
Extraction septum power supply & $<300$ \\
Extraction septum magnet       & $<300$ \\
Other duty factor upgrades     & $1000-3000$ \\
Ramped Correctors              & $<300$ \\
Collimators                    & $<300$ \\
Cogging and Notching           & $<300$ \\
Larger aperture RF cavities    & $3000-10000$ \\
Inductive Inserts              & $<300$ \\
RF for space-charge reduction  & $300-1000$ \\
\hline
{\bf Main Injector} & \\
Additional RF power            & $1000-3000$ \\
Additional magnet power        & $10000-20000$ \\
Cycle time reduction with tuning & $<300$ \\
Dampers                        & $< 300$ \\
Collimators                    & $300-1000$ \\
Barrier RF stacking components & $1000-3000$ \\
\hline
\end {tabular}
\caption { \label {tab:costrange} Approximate ranges
 of cost (including cost for
engineering and technician manpower) for Booster and MI upgrades.}
\end {table}

\begin {table}
\begin {tabular}{|l|rrrrrr|r|}
\hline
Labor Type & \multicolumn {6}{c}{FTEs needed per FY} & Total\\
     & 02 & 03 & 04 & 05 & 06 & 07 & FTE-years \\
\hline
Technician         & 3 & 13 & 19 & 19 & 15 & 7 & 76 \\
Engineer           & 3 & 9 & 9 & 7 & 4 & 2 & 34 \\
Physicist          & 3 & 11 & 11& 7 & 5 & 3& 40 \\
\hline
\end {tabular}
\caption { \label {tab:manpower} The manpower profile and total manpower 
required to undertake the proposed upgrade program.}
\end {table}


\begin {table}
\begin {tabular}{|l|l|l|l|}
\hline
Investment level  &  Very Rough Cost & 120 GeV Protons  & 120 GeV Protons \\
                  &                  & in 2005          & in 2008         \\
\hline
$\approx$None     &  $\approx$ \$0 & $1.3\times10^{20}$ & $1.3\times10^{20}$ \\
Small      &  $\approx$ \$5  M  & $2.8\times10^{20}$ & $3.0\times10^{20}$\\ 
Medium     &  $\approx$ \$15  M  & $4.0\times10^{20}$ & $4.5\times10^{20}$\\
Substantial & $\approx$ \$45M     & $5.0\times10^{20}$ & $8.0\times10^{20}$\\
\hline
\end {tabular}
\caption { \label {tab:protonintensity} The total protons per year which
can be expected to be accelerated to 120 GeV for several different levels
of investment in the existing accelerator complex. It will take time for
some of the improvement programs to be carried out. Numbers are shown for
2005 and 2008 assuming an adiabatic investment. Note that these are the
total protons which are accelerated, some of which go to anti-proton 
production, some to NuMI and a few for other purposes. Note that the
investment levels all include an assumption of twice the current number 
of protons acclerated to 8 GeV in the Booster than currently needed just
for Mini-BooNE operation.} 
\end {table}


The bottom line is that it appears that increasing the proton intensity
within the existing accelerator complex will certainly be possible within
the timescale of 2005-2008. Table \ref {tab:fundingprof} shows an
approximate suggested
funding profile which should be technically achievable given available
manpower and access and which is consistent with the implementation
timescale which we have presented in this report. Because it is beyond the
scope of the current work to produce detailed cost estimates, we provide
these estimates only to offer guidance on the scale of the upgrade project.
It is important that where items are not already part of the current planning
that more detailed cost estimates be developed prior to approval.
Table \ref {tab:costrange} shows approximate cost 
ranges for the various upgrade projects.

We note that some items can deliver increased intensity
at relatively small cost. We recommend that these items should be pursued
as the highest priority. The relatively more expensive investments also
present very attractive improvements in intensity which we think also 
present a good value. We believe that it is of great importance that the
laboratory pursue barrier RF stacking in order to address a potential
strong conflict in its main experimental programs from 2005-2008.
We note that the annual costs of the upgrade program
discussed here are small compared either to the cost of building new 
large-scale detectors or construction of a new accelerator. To give some
feel for how the proton intensity may scale with the overall level of 
investment (most suitably planned for a given total), table 
\ref {tab:protonintensity} lists the total number of protons which we
expect can be acclerated to 120 GeV in 2005 and in 2008 (for all purposes)
at a given level of investment (assuming a smooth funding profile as shown
in table \ref {tab:fundingprof}).  


Table \ref {tab:manpower} shows the 
total FTEs required for technicians, engineers and physicists to carry out
the proposed program of upgrades. As with the costs, the manpower estimates
reported here should be taken only as guidance of the scale of the project
rather than detailed estimates.
We believe that with a combination of
Fermilab and MINOS collaboration manpower that it should be possible to
meet these needs.

We conclude that with a modest investment that Fermilab will be able to 
meet the original planned 
proton intensity for MINOS, even in the light of other
demands for protons such as Mini-BooNE. With larger investment in the
current accelerator complex, but with no fundamental changes, we believe
that it will be possible to even substantially exceed the original
design intensity. We believe that a clear path does exist to meeting
the charge to this committee to ``identify a list of improvements which
appear to have the best chance of delivering a total of $12\times10^{20}$
protons on target for MINOS in a three year period starting in April 2005.''


%Roughly, one might categorize the expected
%improvements based on level of investment and effort for a particular
%proton intensity. Table \ref {tab:protonintensity} summarizes the
%approximate situation that can be expected based on a particular level of
%investment.


\section {Acknowledgements}

In the preparation of this report, we have benefitted greatly from the 
suggestions and advice of many members of Fermilab Beams Division. We wish to
thank all members of Beams Division who have helped and contributed to this
report. We particularly wish to acknowledge the advice and suggestions of
D. Capista, W. Chou, W. Foster, P. Kasper, S. Mishra, K. Ng, 
J. Steimel, R. Webber, D. Wildman, D. Wolff and M. Yang. 
Some of us also take note
that two of our members are retired members of Beams Division (Martin and
Griffin). Their contributions have been invaluable.

\newpage
\appendix 
\section {Characteristics of the current MI 120 GeV beam}

Given the tight limits on allowable losses in the NuMI primary proton 
beam line, it is essential to know the characteristics of the extracted
beam from MI, mainly transverse emittance, momentum spread 
and size of beam halo. The present $\bar p$ stacking cycle, where protons
are accelerated to 120 GeV as in the case of NuMI, is what is available 
to perform these measurements and it has been extensively used.

Transverse emittances have been measured in MI with the Flying Wire system.  
Table \ref {tab:transvemit} summarizes measurements at injection (8.9 GeV) 
and extraction energies (120 GeV) at the highest intensity of 
$4.5 \times 10^{12}$ protons/cycle currently available for the
$\bar p$ stacking cycle.
Variations of the order of 1 $\pi$ mm mrad are usually observed in these 
values, reflecting Booster operating conditions.

Longitudinal emittance measurements have been performed on the circulating 
beam in MI by digitizing the signal from a Resistive Wall Monitor. 
Table \ref {tab:longemit} summarizes the results for a beam intensity of 
$4.5 \times 10^{12}$ protons/cycle. 

The observed increase from injection to extraction of a few $\pi$ mm mrad 
in the horizontal transverse emittance and of about a factor 2 in the 
longitudinal emittance are being investigated. Measurements will be 
repeated after the commissioning of transverse and longitudinal dampers 
in MI and with the Booster dampers operating in steady conditions.  

At least in the first phase NuMI will run concurrently with $\bar p$
stacking: six batches are injected into MI, one batch is first extracted 
to the $\bar p$ source and the remaining ones are extracted to NuMI. 

In the last few ms of the flattop portion of the cycle at 120 GeV, 
bunch rotation is being performed to extract beam to the $\bar p$ source 
with a smaller bunch length. A sudden decrease of the RF voltage for a few
ms causes oscillations in the bunch length at twice the synchrotron
frequency and beam is extracted to the $\bar p$ source at the minimum 
of the bunch length. 

Transverse emittance and momentum spread measurements have 
been performed in these conditions in the P1 line, by recording beam profiles 
on MW's, and about a factor of two increase in momentum spread has been 
observed, as roughly expected. 

Bunch rotation is going to affect all batches in MI, 
and consequently also beam extracted to NuMI. The five batches for NuMI
have to be extracted at the maximum of the bunch length, to provide,
in principle, a reduction in momentum spread.
Preliminary observation of the bunch behaviour show large phase 
oscillations for at least some of the bunches, which would increase 
the momentum spread. More investigations are needed on this issue, in
particular after the commissioning of the longitudinal dampers in MI,
which are supposed to provide a solution to this problem.

Set up of instrumentation for the measurement of beam halo is underway.

Multi-batch operation has been recently resumed achieving an intensity of 
$1.5 \times 10^{13}$ protons with six batches. Measurements of beam 
parameters in these conditions will be performed soon.

\begin {table} [h]
\begin {tabular}{|l|c|c|}
\hline
 & Vert. Emittance & Horz. Emittance \\
 & ($\pi$ mm mrad) & ($\pi$ mm mrad) \\
\hline
Injection & 16 & 17 \\
\hline
Extraction & 16 & 21 \\
\hline
\end{tabular}
\caption {\label {tab:transvemit} Transverse Emittance measurements}
\end {table}

\begin {table} [h]
\begin {tabular}{|c|c|c|c|}
\hline
 \multicolumn {2}{|c|} {Injection} & \multicolumn {2}{|c|} {Extraction} \\
\hline
$\Delta p/p_{95\%} (\%)$ & $\epsilon_{95\%} (eVs)$ & $\Delta p/p_{95\%} (\%)$ 
& $\epsilon_{95\%} (eVs)$\\
\hline
0.20 & 0.17 & 0.079 & 0.36 \\
\hline
\end{tabular}
\caption {\label {tab:longemit} Longitudinal emittance measurements for
Main Injector beams. The $\Delta p/p$ values are without bunch rotation.}
\end {table}

\section {Details on Booster Improvements}
\subsection  {Hardware upgrades to permit faster cycle time}

         The Booster magnets
         cycle at 15 Hz, but not all Booster components are currently able
         to cycle that fast. The current accleration rate is limited to 
         about 3 Hz.
         Several upgrades are already in progress:
         \begin {enumerate}
         \item New extraction septum power supply: should permit 4.5 Hz. 
               Ready soon.
         \item New extraction septum magnet: should permit 7.5 Hz. In
               fabrication. 
         \item As discussed above, additional rate increases, beyond those
               already planned are very attractive. Although the existing
               RF system is nominally designed to operate at 15 Hz, its
               reliability in such a mode is not known and may require
               substantial upgrades \cite {Webber2}. We recommend that 
               upgrades with rate capability approaching 15 Hz should be
               undertaken (this may be staged over several years). New,
               large aperture cavities are already being considered,
               and these should certainly be designed to operate
               at 15 Hz, if built.
         \item The ``ORBUMP'' magnets, which steer the beam through
               the stripping foils during injection, are currently
               limited to 7.5 Hz due to heating.  We recommend that
               plans for new, cooled, magnets proceed on a timescale
               useful to NuMI.
         \end {enumerate}
It is possible that additional upgrades will be identified as the cycle
time is increased and systems may be stressed beyond their nominal
capabilities. Hence, a program for decreasing the cycle time should 
have adequate contingency assigned to handle such circumstances and 
some attempt to explictly estimate this would be useful.

\subsection {Ramped correctors}

While the primary lattice elements of the booster all ramp with
beam momentum, the trim dipoles have historically been operated in
DC.  This means that as the booster cycles, the beam typically
moves both horizontally and vertically.  Also, 
there are certain locations, such as near
collimators, where it's desirable to have positive, time-based
control over the beam throughout the cycle.

For this reason, a set of programmable control cards have been installed in both the
horizontal and vertical planes in order to control the trims currents as
a funciton of time in the acceleration cycle.  A program has been
written to measure the deviations from the beam from an optimal
orbit, as a function of time, and calculate the necessary currents
to correct this.  The program is presently being tested and improved.


\subsection {Beam Collimation}

At the moment, it's believed that there is a substantial halo on the booster
beam, which results in losses which are undesirable, both because they
can occur at high energy and they occur at problematic locations in
terms of tunnel activation and above ground radiation levels.  

In order to better control  protons losses, 
a new collimation system has recently been designed and installed
\cite {Drozhdin}. This system consists of thin carbon primary collimators 
followed by thick copper secondary collimator/energy absorbers. Tests on
the effectiveness of the current systems are underway. It is expected that
the current design should be adequate for running for Mini-BooNE over the
next couple of years. However, for combined MiniBooNE and NuMI running at
ever higher Booster proton intensity it is expected that additional tuning
of the collimation system will be necessary. Before such work can be
undertaken, it will be essential to learn from the performance of the current
system, along with the ramped correcters. We certainly support the current
work which is underway and expect that some future work is likely necessary.

\subsection {Resonant Halo Extraction}

Another idea which has been proposed to work in conjunction with
the collimator system is to resonantly extract the halo.  In this
scheme, the trim quadrupoles are pulsed at some point in cycle to 
move the tune near an instability.  Octupoles are then energized to
move high amplitude particles onto the resonance.  These high amplitude
particles will then become unstable and hopefully be absorbed in the
collimators.  This is the same effect which was used very successfully
used to resonantly extract the Tevatron beam during the fixed target
program.

Initial studies indicate that the technique has promise in the booster.
Successfully implementing it requires the ability to measure the booster
tune as a function of time in the cycle, which cannot currently be
done in the booster.  

Work is being done to implement real time tune measurement in the
booster and we feel that this work, and we feel that this project
should be encouraged, both for the reason stated and under the
general philosophy that better understanding of the booster will
lead to improved performance.


\subsection {Additional RF hardware to spread beam out (reduce space charge)}

   One of the main limitations in accelerating more protons to 8 GeV in the
Booster is that space-charge limitations at injection (400 MeV) cause 
unacceptably high losses to occur as the number of ``turns'' on which 
protons are injected from the Linac are increased. At present, the limits
from proton losses require that no more than about $7 \times 10^{12}$
protons be injected (of which $\sim 2-3 \times 10^{12}$ are lost prior to
extraction).

One relatively simple and attractive way to reduce the
space-charge effects is to spread the beam out more longitudinally
during injection. This
can be accomplished with the addition of another relatively low power RF
system which will effectively produce wider total longitudinal focussing
buckets. Since the space-charge effects are only at the low injection
energies, it isn't necessary for this to operate up to high energies, but
only near injection energies. It also isn't necessary for this RF system
to provide significant acceleration power as most of the RF cavities do. 
However, it must smoothly ``decouple'' as the beam accelerates. 

Implementation of this system requires construction of an additional RF cavity
which would run at about 80 MHz with 10\% of the power of the other RF
cavities,
power and control system. Although the total cost in these components is not
very large (we estimate the total cost for parts and manpower less than
about \$1M), it is a relatively tricky system that will require significant
design and tuning work. The starting point will be a complete ESME simulation.
We anticipate that much of the design and tuning work can be done by
physicists.

\subsection {Inductive inserts}
   Inductive inserts have been proposed by Griffin \cite {GriffinII} to
provide passive compensation for space-charge effects. Inductive inserts 
are simple ferrite tubes which are heated so that the material properties
are just right so that space-charge effects induce a longitudinal 
self-focussing. The principal has been demonstrated in a storage ring at
Los Alamos and has been tested at Fermilab to show no obvious deleterious
effects (there was some concern that it might due to a 74 MHz resonance).
However, the Fermilab test used too little Ferrite to be effective
against space-charge beam blowup. In order to provide enough of an effect,
roughly 12 m total of ferrite tubes would need to be placed in the Booster.
One issue is whether adequate space is available. The ferrite needs to
be of high quality, but the cost is not particularly high, perhaps a few
thousand dollars per meter. Hence, most of the work and cost will be in
studying how to implement this system, the qualification of ferrites and
the insertion into the Booster. The total cost will likely be less than
about \$300k.

\subsection {``Gamma-t'' ($\gamma_t$) System}

The booster is equipped with a set of pulsed quadroples whose purpose
is shift the transition energy downward slightly.  If these are 
pulsed just as the beam is entering transition, it will effectively
cause the beam to ``jump'' through transition.

The system has been tested in the past and has proven quite
effective at preserving longitudinal emmittance through transition.
Unfortunatly, low emittance, high intensity bunches excite
couple bunch instabilities in the accelerating cavities.

For this reason, the ``Gamma-t''  system is not 
typically used, and the emittance is {\em allowed} to blow
up slightly at transion.

It's possible that in the presence of the new, improved
damping system, a compromise can be found between reduced
emittances and bunch stablity.

\subsection {Larger Aperture RF cavities}

A development program is underway to fabricate a new, larger aperture Booster 
RF cavity.  The motivation for this is primarily to reduce the losses at the 
location of the cavities, since these 
require the most hands-on maintenance.  (Present dose levels range from tens 
to hundreds of mrem/hr at one foot.)  The present cavities have an aperture 
of 2.25'', whereas the new cavities will increase this to 5''.  The first
(prototype) cavity will reuse the existing tuners, but subsequent cavities 
would have new tuners along with the outer shell and drift tube.

The plan to fabricate new cavities as opposed to retrofitting the existing 
cavities is based on three considerations.  First, retrofitting old cavities 
would involve a great deal of hands-on labor on cavity components that will 
be quite highly activated.
Second, there is substantial concern about the longevity of 
portions that would not be reworked, in particular the water passages, given 
the already extended usage these cavities have seen.  And lastly, retrofitting
the existing cavities would require replacing two to four cavities 
during each shutdown, and retrofitting them prior to the next shutdown.  This 
process would take too many years to accomplish the goal of replacing all 
cavities within the first year or two of NuMI operation.  

The goal is to fabricate the prototype cavity by early calendar year 2003, 
test it extensively in the first half of the year, and install it in the 
Booster in the summer shutdown.  A few weeks of in-situ operation should 
confirm if it is acceptable to allow ordering of parts for building an entire 
set of new cavities at the beginning of FY04.  Completion of fabrication 
would be in FY05, followed by testing and installation.

The estimated cost for new cavities is approximately \$5-6M.  To meet the above
schedule, it is imperative that adequate resources be devoted to this effort, 
including technicians (approximately 4), a mechanical engineer and an rf 
engineer.  The level of effort of each of these is more than half-time.  One 
of the major efforts will be to qualify ferrites.  This can go on in parallel 
with the prototype fabrication.  

The goal of installing new cavities with larger apertures is first to reduce 
losses in the areas requiring the most hands-on maintenance.  Reduction of 
losses alone should increase Booster performance by a few percent.  But 
removing the losses from the high-maintenance areas should allow pushing the 
intensity considerably higher than if the losses remain at these locations.  
While difficult to quantify, the gains could easily be another 10 to 20 
percent.

Clearly, these new cavities should be designed to operate at the
full 15 Hz maximum booster operation.

\subsection {Extraction Timing Issues (``Beam Cogging'')}

Early in the booster cycle, a kicker is used to remove about
5\% of the beam, creating a ``notch''.  Extraction is timed
so that this notch is passing the extraction septum while 
it is ramping up.  This dramatically reduces extraction losses
by preventing the beam from sweeping over the septum.

There are a fixed number of Booster revolutions between the
creation of the notch and beam extraction, but for reasons
that are not entirely understood, the total {\em time} it takes
to make these revolutions varies from cycle to cycle on the order
of several microseconds.  

At the moment, this is not a problem because the Booster extraction
determines the precise time for transfer into the Main Injector; however,
obviously this scheme will not work when we go to any sort of multibatch
injection.  Some method will have to be found for fixing the time
between the creation of the notch and the booster extraction.  There
have been some ideas successfully tested, they have never been
demonstrated at high intensity.

It's important that work go forward to find a scheme to properly
cog the Booster to the Main Injector, and hopefully to understand
the mechanism which leads to these time variations.

It should be noted that this is an area of Booster performance 
which is of no concern to MiniBooNE and so will not likely 
receive a great deal of attention without a push from NuMI.

\subsection {Improvements needed for Barrier RF stacking in MI}

  In order for Barrier RF stacking of Booster beams into the Main Injector
to work efficiently, the longitudinal emittance of beams from the Booster must
be reduced compared to current operation. With the existing damper systems
working, the longitudinal emittance of the beam extracted from the Booster
for a total of $4.5 \times 10^{12}$ protons in a batch has been measured to
be 0.15 eV-s. The main requirement for barrier RF stacking is a that the
momentum spread of beam from the Booster when injected into the Main Injector
should not be more than about 5 MeV. With appropriate bunch treatment, it is
expected that this can be achieved if the longitudinal emittance
of the Booster beam is 0.1 eV-s \cite {Ng}. 

There is a trade-off in the Booster between the emittance and the total 
number of protons accelerated per batch as one approaches the maximum
intensity. This is simply due to the fact that at high intensity, instabilities
cause the beam losses to go up and these are what determine the maximum
intensity. Hence, with the new ramped correctors and dampers 
that are already being 
installed and commissioned, it may be possible to find an
operating condition with $>4\times10^{12}$ protons per batch that will meet
the longitudinal emittance requirements for barrier RF stacking. This will
be interesting to test over the next few months. However, it is also possible
that new systems will be needed for this purpose. One possibility is
the implementation of additional RF controls (along with a new cavity?) which
would permit phase rotation of the beam after acceleration to 8 GeV but
before transfer to the Main Injector. Other improved RF control systems 
may also prove worthwhile. Finally, the larger aperture RF cavities mentioned
above could prove very important at higher proton intensities for
delivering beams with the required emittance.

\section{Details on Main Injector Improvements}

\subsection {Additional RF power and voltage for higher intensity}
\label {sec:MIRF}

\subsubsection {Operating scenario and parameters}

It is assumed here that in order to reach desired NUMI Main Injector beam the 
technique of longitudinal barrier stacking will be implemented so that twelve 
Booster batches may be injected and accelerated on each MI cycle for total 
beam intensity $6 \times 10^{13}$ protons per cycle.
With the Booster operating at the present 15 Hz rate, the total 
injection time will be ~0.95 s (including 10-20 ms for adiabatic capture of 
the de-bunched beam in the MI following barrier stacking).  Each Booster 
batch will contain $5 \times 10^{12}$ protons in $\approx 82$ (of 84) 
adjacent buckets, with longitudinal emittance 0.1 eV-s per bunch (which
following all barrier stacking manipulations will result in a 
stacked longitudinal emittance of about 0.5 eV-s in the MI). The steady 
state dc beam current (during the passage of adjacent bunches), is 
$\approx0.52$ A. In 
order to decrease the MI cycle time and offset somewhat the increased 
injection time required by barrier stacking, it is assumed that the maximum 
MI ramp rate may be increased from the present 240 GeV/s to 260 GeV/s 
      
      The Main Injector RF system is assumed to contain  the eighteen 
existing RF cavities, with a single Y567B (4CW150,000) power amplifier tube 
installed in each cavity. The cavities are assumed to have $R/Q = 120$ and 
$Q = 6500$ (at frequencies away from injection), 
giving $R_{sh}=7.8 \times 10^5$ Ohms.  The 
cavities are expected to operate with maximum accelerating voltage near 
240 kV each, with voltage step-up ratio from anode to gap 12.25:1.
      
      We assume that ``local'' amplitude and phase feedback systems with 
bandwidth substantially larger than the synchrotron phase oscillation 
frequency are installed and operative so that the cavities may be detuned such 
that the power amplifier load appears 'real' without concern for Robinson 
stability or bucket area reduction factor.  (The presently proposed amplitude 
control system will probably be adequate.  Additional phase feedback may be 
required as the cavity tuning system may not have sufficient bandwidth.)  
Also a relatively fast feed-forward system will be required to prevent rapid 
excursions of the RF phase and amplitude during gaps in the bunch train.

 We expect that changes will be required in ancillary equipment, such as solid 
state RF drive amplifiers, series tube modulators, anode or screen grid power 
supplies, or feedback systems, when existing systems are found not to be 
adequate to allow the RF cavity, with its existing power amplifier tube, 
to reach maximum power capability.


For constant acceleration rate and RF voltage, the generated bucket area 
reaches a minimum at $3^2 \gamma_t$. Voltage, bucket area and power 
calculations are examined at 39 GeV/c. 

\subsubsection {RF Voltage and Bucket Area} 

For acceleration at 260 GeV/s the required accelerating voltage, 
$V_{ac}\sin \nu_s$ is:
\begin {equation}
V \sin(\phi_s) = \frac{2\pi R}{c} \frac {d(pc)}{dt} = \frac {260 \times 10^9}
{90.314 \times 10^3} = 2.88 \times 10^6 V.
\end {equation} 

With maximum cavity voltage, 240 kV, eighteen cavities generate 4.32 MV.  
The synchronous phase angle $\phi_s$ is 41.8 deg. and the 
'moving bucket factor' $\alpha(\Gamma) =  0.195$,  
($\Gamma \equiv \sin \phi_s$). The bucket area is
\begin {equation}
A_b = \alpha(\Gamma)\frac{8R}{hc}(\frac{2E_sV_{\rm RF}}{\pi h \eta})^{1/2} =
0.195 \times 8.1 = 1.6 eVs.
\end {equation}

\subsubsection {RF Beam Power and Stability Considerations}

The RF beam power required is
\begin {displaymath}
P=\frac {e\beta cV\sin\phi_s}{2\pi R} = 4.32 \times 10^{-8}
\end {displaymath}
watts per proton, or 
2.6 MW at the design beam intensity (144 kW per cavity with 18 cavities).  In 
this operating mode the tetrode anode dissipation is 146 kW and the cavity 
dissipation is 39 kW for total RF power delivered per amplifier 327 kW.  The 
average tube cathode current is 24.7 A.  All operating parameters of the 
amplifier tube are within the rated maximum values.  

The total power dissipated in the tube anode and the RF cavity is slightly 
larger than that delivered to the beam. One of the Robinson stability 
conditions is that this ratio exceed unity.  In this case the margin for 
stability adequate, but not 
great.  It is assumed that an additional stability margin will be maintained 
through the several feedback systems installed.  If those prove inadequate, 
stability can be enhanced by the installation of additional water-cooled RF
loads on each cavity.  The amplifier design is such that substantially more 
power can be developed by each of the tetrodes by the addition of larger 
cathode drive amplifiers and changes in the control and screen grid voltages. 
Such an addition would also require installation of additional dc power sources
to the system.  

\subsubsection {Conclusions}

Operation of the existing Main Injector RF system at beam intensity 
$6 \times 10^{13}$ 
protons per cycle with minor modifications is described.  The primary caveat 
appears to be a voltage limitation, not a power limitation.  The 1.6 eV-s RF 
bucket 
area described above is barely adequate to contain bunches with emittance 0.5 
eV-s. Such bunches may be anticipated with slightly improved Booster 
performance and very good barrier stacking performance.  Additional 
longitudinal dilution may occur during transition crossing in the Main 
Injector. Several additional Main Injector RF cavities exist and could be 
installed and placed in operation with minimal effort. 

\subsection {Additional RF power and Magnet Power for faster cycle time}
\label{sec:MIcycleramp}
   The rate at which protons can be accelerated in the Main Injector is
determined in part 
by the rate at which RF power can provide the necessary energy,
the RF voltage which is required to accelerate the protons in a given number
of revolutions and the rate at which the bend and quadrupole magnets can be
ramped up and then back down at the end of the cycle. Assuming that all other
tuning of the system (discussed in section \ref {sec:MIcycletune}) has been
completed, any additional improvements in cycle time will require reduction
in the ``ramp times''. The current MI ramp times are primarily determined by
the total magnet power supplies. The present rate at which the MI can be
ramped (well, almost as discussed below) is about 0.5 seconds each for the ramp
up and ramp-down times. Combined with all other features of the cycle this
determines the overall minimum cycle time of 1.5 seconds. In a recent study
(part of the proton driver upgrade study) the possibility of reducing the
total cycle time to 1.0 seconds has been studied, with most of the gain
coming from being able to ramp the magnets faster than the current power 
supplies permit \cite {PDstudy, Wolff, Mishra}. The needs for additional
RF power are mostly met within the envelope discussed in this document.
(However the combination of both higher ramp rate and stacking may require
more RF power than discussed here. We have not yet had time to consider
the implications of both.)

According to the study presented by Wolff \cite {Wolff}, relatively small
improvments are necessary in both the bend magnet power supply bus and the
quad power supply bus in order to provide reliable operation at 1.5 s cycle
time. On the order of \$200k is required for additional transformers and
power supply components in each of these systems. No significant modifications
will be needed in either system in terms of bus components, total power, or
layout of the system. 

For decreasing the cycle time to 1.0 s, substantial new investment in power
supply components is necessary. The basic
idea is to double the maximum power supply voltage in order to halve the
time of the ramps. The nominal plan would add two additional bend supplies
and one quad supply to every MI service building. This would increase the
bend bus voltage to ground from 500 V to 1000 V. This may prove untenable 
in which case it would be necessary to increase the number of feed lines
between the power supplies and the magnets. With the exception of MI 60,
it appears that there is adequate duct space to accomplish this without new
civil construction. Additional transformers and other components will also
be necessary at the Kautz road substation. Some civil construction may be
required to provide additional space in the MI service buildings. A rough
cost estimate exercise has suggested that the total cost of these upgrades
are in the neighborhood of \$20M. It is clear that this will need several
years to carry out all of the work and acquire all the components.

Although this is certainly the most expensive single upgrade option which 
we discuss in this document, we do not believe that it should be dismissed
simply because of the relatively high cost. We note that the investments
made here will provide better performance even at the time that a new
proton driver would become available. Running the Main Injector at 1.0 Hz
rather than 1.5 Hz given this cost will always be a relatively good bargain
compared to increasing the statistics of neutrino events by building a
detector which is 50\% larger. Hence, although we realize that the relatively
large cost of this upgrade will require it to be carried out over several
years, we strongly recommend that it be considered as part of the total
upgrade planning so that it can be taken into account in the other system
designs and so that detailed planning work and acquisitions can get underway.

\subsection{Reduction in cycle time (other than from more RF and magnet power)}

\label {sec:MIcycletune}

The Main Injector cycle length for 6 batch operation is presently 1.867 s,
with a maximum ramp rate of 240 GeV/s. One way to decrease the cycle time
is to increase the ramp rate. This is discussed in section 
\ref {sec:MIcycleramp}. However, there are relatively more subtle modifications
that may also permit a significant reduction in the cycle time.
Careful adjustments
of the various parts of the ramp as conservatively designed now might lead
to a gain of about 100 ms and perhaps as much as 190 ms (5-10\%).
There are three main components of the ramp where it may be possible to
reduce the current cycle time without significant invesment in new hardware.
\begin {enumerate}
\item The set of parabolas from 85 to 120 GeV: Optimizing these
could result in a cycle time reduction of about 50 ms. It is necessary to
demonstrate that this does not adversely affect the beam.
\item The flattop portion: In this time, the final RF frequency 
adjustment, bunch rotation and extraction are performed. It may be possible
to reduce this to about 50 ms from the present 70 ms.
\item The reset part of the ramp: In this time, magnets are ramped down to 
6.7 GeV to minimize the hysteretic contribution. Currently this takes 125 ms.
Studies are necessary to determine how much this might be reduced. We note
that at one time the Main Ring operated with no reset in the ramp. Instead,
a different ramp cycle was used depending on the current hysteretic state.
It is expected that at least 30 ms might be trimmed from this reset time and
with careful attention to hysteretic condition it may be possible to reduce
it even more. 
\end {enumerate}
Some initial studies are already planned regarding cycle time reduction \cite
{Capista}. In order to achieve the full reduction from this kind of tuning,
additional performance studies will be necessary.

\subsection {Dampers}
The MI Dampers are a series of pickups and kickers which are used to stabilize
the beam. The pickups get the signal from 
the circulating beam about its energy, position and motion 
of the beam as it goes
through the accelerator. The analog information form the signal is used to
provide feedback to the beam through the kickers.
There are three damper systems necessary for controlling instabilities in the 
Main Injector:
\begin {enumerate}
\item {\bf Longitudinal system:}
Controls the beam energy and the time oscillation 
about some fixed reference. Most longitudinal focussing and control is 
provided by the high level RF system with the cavities and power amplifiers. 
The damper system provides perturbations on the main high level RF control 
signals.The timing and bunch length are measured 
with a stripline pickup located at MI-60, and the processing occurs in the
MI-60 control room. The signals are then applied to the low level and high 
level RF controls.
\item {\bf Narrowband transverse system:}
Controls the variations in the transverse position 
of the beam relative to some fixed orbit resulting from 
transverse coupled bunch mode instabilities only 
closest to the fundamental mode.
\item {\bf Wideband transverse system:}
Controls the variations in the radial position 
of the beam relative to some fixed orbit resulting from 
transverse coupled bunch mode instabilities. The wideband damper system 
corrects instabilities related with betatron lines closest to the
fundamentals but also around other revolution harmonics
\end {enumerate}


When the MI was first commissioned in 1999, both the longitudinal and 
narrowband transverse dampers existed. The wideband transverse dampers have
not yet ever been deployed.
This permitted operation with six batches 
and a total of $2\times10^{13}$ protons. At present no damper system 
is in use and in fact some of the control electronics for these systems 
has been removed for use elsewhere in the accelerator complex. However,
the components installed in the MI ring are still there. New control
hardware for the longitudinal damper systems exists but needs to be 
assembled and commissioned. The longitudinal damper can be 
recommissioned within a few months with part time effort of a physicist,
engineer and technicians.

The existing pickups can be used to recommission the narrowband transverse 
system. It will be necessary to duplicate some low level electronics at a 
cost of about
\$10K. A high power RF switch will be needed to use the same damper 
alternatively between the proton and the pbar cycle. The high power RF switch 
is expected to cost about \$15k but an appropriate commercial switch has not 
yet been identified. Assembling and commissioning this system
will require a team comprised of a physicist, an engineer and a technician
working for about six weeks. An alternative is to use separate power 
amplifiers and kickers for narrow band transverse dampers for proton and pbar 
damping. In that case, it will be necessary to build new kickers and 
order power amplifiers which will increase the time before this system can
be implemented. This will cost about \$65k.

The MI group is interested in commissioning a new digital wideband damper 
system which will be used to correct the instabilities related with betatron
lines not only closest to the fundamentals but also around other revolution
harmonics. The new system will analyse the data 
from pickups for instabilities in several modes, do a digital correction and 
supply the feedback for correcting several modes at the same time. This system
is not yet known to be needed but may become necessary as the proton intensity
is increased.

Existing pickups, kickers and power amplifiers can be used as inputs to
this system. New processors and related hardware needs to be procured and
will cost at least \$100k. A team consisting of a physicist,
an engineer and a technician can accomplish this within a year. The MI 
department expects to begin work on the damper system this year and 
expects to commission it in the spring/summer of 2003. 

With all planned dampers in operation, and with proper tuning of the MI,
it is expected that a total of $2.5-3.0\times10^{13}$ protons per cycle
can be accelerated to
120 GeV. Additional information can be obtained from reference 
\cite {Steimel}. It is anticipated that the capabilities of the 
currently envisioned wide-band
transverse damper will permit the operation to exceed $3.0 \times 10^{13}$
protons per cycle. 

\subsection {Collimators}
   It is not yet clear whether any Main Injector collimators will be necessary 
for the proton intensities described here. For intensities as high as those
discussed with a new proton source, it appears that such collimators will be
needed in order to protect the components of the Main Injector from too much
radiation damage due to even relatively small proton losses. These are 
described in the recent proton driver upgrade study \cite {PDstudy}. It
is our expectation that such collimators will not be necessary for the
intensities discussed here.

\subsection {Barrier RF stacking}
\label{sec:barrierRFstack}

   At present in the Main Injector, all longitudinal focussing and acceleration
is accomplished using the standard 53 MHz RF buckets. This works fine for
normal injection and acceleration. However, if one wishes to do more complex
operations, such as stacking more than six batches of protons, one needs 
another tool. In ``slip stacking'', two batches of protons from the Booster
are injected at slightly different momenta, but both still contained within
the acceptance of the Main Injector. One then waits $\sim 100$ ms for the
beams to coalesce due to the slightly different velocities. It is not clear
that this will work for more than just two batches of protons and it is
unclear how efficient it will be even for two batches at high intensity.

Barrier RF stacking of Booster protons into the Main Injector was first
proposed by Griffin \cite {Griffin}. A recent study has been carried out
by Chou, Ng and others which further develops the concept \cite {Ng}.
In Barrier RF
stacking, batches of protons are all injected into the Main Injector at a fixed
momentum but then a ``travelling'' RF square wave slightly accelerates each 
batch (only once) to a higher momentum and moves its location in the Main
Injector by half a batch length. This permits a second Booster batch to
be injected with an offset of 1/2 batch length to the first (and so on)
up to a total of 12 batches.
In fact, it is believed that this can be done within the 70 ms 
period of the Booster so that injections could occur at the full possible
rate of 15 Hz (with appropriate Booster upgrades). Nominally, this will
permit the MI to be filled with a total of 12 batches of protons from the
Booster in 800 ms compared to 6 batches in 400 ms with no stacking. If
we then assume that the MI cycle adds 1.5 s to these numbers the resulting
increase in the rate of protons accelerated to 120 GeV is a factor of 1.65.
Assuming that one ``stacked batch'' is extracted for pbar production, the
rate at which protons are delivered for pbar production will be about the
same as with slip stacking. The important difference is that the number
of protons delivered for NuMI will go up significantly at the same time
as the protons for pbar production are increased. 

With slip stacking,
it is possible that the intensity to NuMI could be decreased by 10\%. Hence,
the relative difference for NuMI with barrier RF stacking compared to
slip stacking for the collider is a factor of $1.8$. If it is possible to
efficiently slip-stack in multi-batch operation, then the ultimate
difference may not be as large as this nominal factor. However, 
we believe that the potential for a dramatic difference for NuMI, 
given the same performance
for pbar production makes a rather strong argument in favor of pursuing
RF barrier stacking as the highest priority stacking program. 
The goal should be to start to use RF barrier stacking in 2005. (It is
possible that some use of slip stacking in the interceding years may prove
effective for pbar production or even that some mixed use in future
years could provide a best overall optimization of the program.)

The principles of barrier RF stacking in the Main Injector are already under
demonstration in the Recycler \cite {Recycler}. 
In fact, the implementation in the recycler is
more complex than needed for the Main Injector. The main requirements for
barrier RF stacking in the Main Injector are:
\begin {enumerate}
\item Beams from the Booster with sufficiently small longitudinal emittance.
      It is estimated that an emittance of 0.1 eV-s should be acceptable.
      Even better emittances may help keep proton losses down and total
      efficiencies high. Two issues drive the need for small emittance.
      The first issue is that the momentum spread must be no more than
      about 5 MeV in order to avoid proton losses during the stacking
      process. The second issue is to limit losses at transition
      in the Main Injector. It
      is estimated that barrier stacking will increase whatever emittance
      comes from the Booster by at least a factor of 1.6, on top of
      the double width from stacking. Hence, the total anticipated
      longitudinal emittance of the beams in MI after stacking has been
      estimated to be 0.47 eV-s \cite {Ng}. It may be possible
      to compensate for this using improved techniques for transition
      crossing. However, we note that this nominally should already be
      acceptable in the Main Injector \cite{MIDesign}, though it is 
      generally not actually
      the normal operating mode at this time. This is something that could
      be tested in the near future.
\item Sufficient RF power in the Main Injector to handle the higher beam
      intensities and sufficient RF voltage to handle the larger longitudinal
      momentum spread of the stacked beams. As discussed in section 
      \ref {sec:MIRF}, we believe that this can be achieved relatively
      easily. However, this is also one of the more interesting things
      which should be experimentally demonstrated and argues for an early
      launch of a test of the barrier RF stacking in the Main Injector.
\item Implementation of barrier RF cavities and power and control
      system in the Main Injector. Studies must be done to understand
      exactly the number of cavities required. It is perhaps as few as two
      cavities and as many as four. This is in fact nominally all the new
      hardware which needs to be built. This requires several FTE-years
      including physicists, engineers and technicians to implement in
      period of about two years at a cost between \$1-3M.      
\item Ability to accelerate bursts of 12 batches of protons at the full
      rate of 15 Hz from the Booster, every 2.267 seconds (or faster if the
      MI cycle time is reduced). Although one can implement barrier RF
      stacking without this, the relative efficiency is improved with the
      highest acceleration rate from the Booster. We note that with Mini-BooNE
      running at the same time that this will mean an average rate $>11$ Hz.
\item Efficient adiabatic capture of the debunched stacked beams. Unlike
      slip stacking, the barrier stacked beams lose all memory of the 53 MHz
      structure. Hence, the 53 MHz RF must be applied slowly once stacking
      is complete or a large fraction of protons will be lost. It is 
      estimated that if the adiabatic capture is done over a period of about
      10 ms that 3\% of the injected protons will be lost \cite {Ng}. 
      It may be best
      if it is arranged that these are lost in a dedicated collimation system.
      However, studies on this feature need to be done to better quantify
      the realistic capture.
\end {enumerate}
Due to the relatively spread-out beams, it is expected that space-charge
effects will not be a limiting factor for the barrier
stacked beams. It remains to
be seen whether there will be additional new stability control issues with
beams of this type and at high intensity in the Main Injector.

\subsection {Proton safety envelope}

   As with the Booster, the Main Injector has a set of administrative 
operations constraints to assure that the integrated observed radioactivity 
in and above the tunnel is not too large, even under accident conditions. 
The current administrative limits
are:
\begin {enumerate}
\item $<5.7\times10^{16}$ protons per hour at 8 GeV.
\item $<3.9\times10^{16}$ protons per hour at 120 GeV.
\item $<3.3\times10^{16}$ protons per hour at 150 GeV.
\end {enumerate}
These numbers have been set very conservatively for the start of MI operations
and should be easily modified upwards following a necessary set of safety
review procedures and measurements. We note that the nominal operations
required to deliver all of the planned protons for Run IIB plus NuMI 
corresponds to about $8.3\times10^{16}$ protons per hour at 120 GeV, more
than twice the current administrative limit. However, those familiar with
the design and initial shielding have suggested that no additional shielding
needs are expected up to at least $10^{17}$ protons per hour at 120 GeV and
the real limits may in fact be well beyond that. We note that $10^{17}$
protons per hour corresponds to the nominal maximum intensity which we
discuss in this document.

Hence, the main impact of this limitation will simply be that 
work will be required along with careful monitoring to establish this new
operational standard. In addition, should proton losses be higher and/or
even higher intensities delivered it may be necessary to take additional
measures to control losses, collimate losses, add shielding or some
combination of these approaches. Again, a program of careful measurements
will be needed in order to determine an ultimate safe operating limit.

\section{Issues in the NuMI Beamline}

It was realized early on that intensity is a crucial concern
in the design 
of the NuMI primary proton beam. Any losses which occur in a line of this 
intensity cause problems, first in irradiating components which may later need 
to be handled or worked on, and second, in leading to groundwater activation. 
The intensity assumed in design of the line is 
$4 \times 10^{13}$ protons every 1.9 
seconds. A significant feature of the design, a somewhat new concept, is a 
Permit System, which has the ability to abort further beam pulses
if any single pulse shows unacceptable losses. 
In addition this system has the ability to inhibit any 
pulse before it is extracted
if any beamline parameters are out of tolerance 
immediately before extraction. This philosophy, of limiting the number of 
unacceptable pulses to at most one, would continue to be viable for beam 
intensities as high as $10^{14}$
 protons per pulse; it would also continue to function 
well if the pulse rate were to increase.

Another subsystem which will be brought on line for NuMI is the Beam Loss 
Budget Monitor. In general Fermilab accelerators and beamlines are approved 
to run a certain amount of beam per hour, week or year. There will be similar 
restrictions, which are enforced administratively via the Beam Budget 
Monitor, on NuMI during commissioning. The rationale for limiting beam to 
any region is to assure that chronic losses in that region are kept below some 
level. In NuMI it is intended to monitor the losses themselves, and to impose 
an administrative limit on the amount of loss observed. Clearly if there are 
chronic losses in the beamline, the loss budget will be exhausted more quickly 
during high intensity running. Thus it is essential that both episodes
and chronic losses be 
kept to an absolute minimum.

For most beamlines a prime consideration for losses is the possibility of 
prompt radiation near the surface. However for NuMI this is not the case. The 
downward slope of the line assures that after beam has left the MI60 region it 
is deep enough that surface radiation is not a serious consideration. While 
the beam is in the region of MI60 and the NuMI Stub it is covered by the MI 
berm.

A concern has been voiced that as the MI intensity rises its beam quality will 
deteriorate to some extent. The relevant measures of beam quality 
are the transverse emittance and momentum spread. If the transverse 
emittance were to grow too large the beam might be too spread out to be 
efficiently 
transmitted by the NuMI line and losses would ensue. Similarly, since there is 
dispersion in the line caused by the large vertical bends encountered, the 
momentum spread manifests itself as a growth in transverse beam size. Indeed 
it was shown convincingly that, for the line as originally designed, momentum 
spread as anticipated for high intensities could not be transported. 

The recent redesign of the line has increased the acceptance in both emittance 
and momentum spread. These acceptances are now greater than that of the MI 
itself \cite {numiprimarybeam}. We endorse this modification to the beamline
design and note that it is critical to future high-intensity operation.

There are still a few areas of concern in the current design for high
intensity operation. A matter of particular concern is the third 
extraction Lambertson which appears to be too close to the beam in the 
current design. Additional studies are being performed to address this issue
and we endorse the need for this additional work. The second concern is in 
the vertical plane at the trim magnet near station 230 m. This trim will be 
replaced by one with a larger aperture if operating experience indicates that 
there are significant losses here. Lastly there is a problem in both planes 
with the aperture at station 350 m. However this aperture is that of a 
collimator, which will be installed at that location to provide protection 
for the focusing horns from misbehaved beam. 

We have discussed the possibility of inclusion of a collimator in the
region between the Main Injector and the main vertical down-bend. It appears
that the current beamline design will not require such a collimator with
forseen operational intensities. However, we believe that prudence call for
keeping such an option open for future upgrades.

\section {Charge to the NuMI Proton Intensity Working Group}

The working group is charged with advising the Directorate and the MINOS 
spokesperson on the number of protons per year that the MINOS experiment can 
reasonably expect to have targeted and actions which can be taken to help 
maximize the total number of protons delivered in a three-year running period.

This advice should be based upon the following:
\begin {enumerate}
\item Document the present capability of the accelerator complex with respect 
to protons per cycle that can be accelerated to 120 GeV in the Main Injector 
in the mixed-mode expected for joint NuMI + pbar production operation.  
Document the beam emittance, both transverse and longitudinal, at 120 GeV, 
and the Booster losses per proton relative to the trip point of the 
interlocked detectors.  The emittances and the losses are functions of 
intensity, so the above measurements need to be done over a range of 
intensities.
\item Document the number of protons per hour that can be accelerated in the 
Booster for the above operating cycle while staying within the safety envelope.
\item Document the number of hours per week that beam can expected to be 
available from the Main Injector.
\end {enumerate}

Based on the above measurements, develop a plan of improvements, ordered in 
priority to the extent possible, that appear most attractive towards increasing
the projected proton intensity per year.  Assuming these improvements are 
implemented, what is the expected gain?  Although it should not be taken as a 
limit, the working group should specifically identify a list of improvements 
which appear to have the best chance of delivering a total of 
$12\times10^{20}$ protons on 
target for MINOS over a three year period starting in April 2005.

Where possible, the working group should identify specific manpower needs, 
from both inside and outside of Fermilab, in order to meet the suggested 
improvement goals.

A final report should be submitted by April 15, 2002.  The working group should
report at each MINOS collaboration meeting and NuMI PMG meeting until then. 


\begin {thebibliography}{20}
\bibitem {NuMITDR} ``The NuMI Project TDR'', Nov. 1998.
\bibitem {Peoples} October 1998 Letter from the Fermilab Director.
\bibitem {RunIIB} ``Plans for Tevatron Run IIB'', Edited by D. McGinnis 
and H. Montomery, Dec. 2001, 
http://www-bdnew.fnal.gov/pbar/run2b/Documents/TDR/Default.htm.
\bibitem {BDGoal} Current Beams Division Goals as Stated in April Directors
Review of the NuMI Beamline.
\bibitem {Webber} R. Webber ``Challenges to The Fermilab Linac and Booster
Accelerators'', Fermilab-Conf-01-121, Presented at IEEE Particle Accelerator 
Conference, Chicago, 2001.
\bibitem {Webber2} R. Webber, ``Report on Booster High Pulse Repetition
Rate Tests'', Fermilab TM-2074, March, 1999.
\bibitem {Webber3} R. Webber, ``Fermilab Booster operational status: Beam
loss and collimation'', Fermilab-Conf-02/106-E, June, 2002.
\bibitem {Drozhdin} A. Drozhdin etal., ``Beam loss, residual radiation, and
collimation and shielding in the Fermilab Booster'', Fermilab-CONF-01-141,
July, 2001.
\bibitem {GriffinII} J. Griffin, Talk at ICFA High Intensity Hadron Beams 
Workshop, Fermilab, April 2002. Also TM notes...
\bibitem {PDstudy} Study for a new proton driver. Final report to be issued.
\bibitem {Wolff} ``MI PS Ramp Rates'', Talk by Dan Wolff as part of the
proton driver study, Feb. 2002.
\bibitem {Mishra} S. Mishra, ``Status of the accelerator complex and  
possible upgrade plans'', Talk presented at the Workshop on New Initiatives
for the NuMI Neutrino Beam, May 2002.
\bibitem {Griffin} J. Griffin, ``Momentum stacking in the MI using longitudinal
barriers'',1998.
\bibitem {Ng} K.Y. Ng, ``Continuous multiple injections at the Main 
        Injector'', FN-715. Also presented at ICFA High Intensity Hadron
        Beams Workshop, April 2002, Fermilab.
        Published in Phys. Rev. ST AB, Issue 6, June 2002.
        Available at http://prst-ab.aps.org/abstract/PRSTAB/v5/i6/e061002.
\bibitem {Recycler} Various documents on the Fermilab Recycler, See for 
instance: ``The Fermilab Recycler: An 8 GeV permanent magnet storage ring with
electron cooling:, J. MacLachlan, Fermilab-Pub-98/388,
 ``The Fermilab Recycler Ring'', M. Hu, Fermilab-Conf-01/187-E, 
July, 2001... 
\bibitem {MIDesign} ``Main Injector Design Handbook''.
\bibitem {Capista} D. Capista, Private Communication.
\bibitem {Steimel}
James M Steimel "http://cns40.fnal.gov/tevatron/steimel/favorite.htm".
\bibitem {numiprimarybeam} J. Johnstone,
``A modular optics design for the NuMI beamline'', Fermilab TM-2174,
April 2002.


\end {thebibliography}

\end {document}
